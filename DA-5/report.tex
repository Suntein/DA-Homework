\documentclass[UTF8]{ctexart}
\usepackage{amsmath}
\usepackage{listings}
\usepackage{geometry}
\geometry{a4paper, margin=1in}

\title{Binary Search Tree 测试报告}
\author{}
\date{}

\begin{document}

\maketitle

\section{概述}

本文档旨在说明 \texttt{BinarySearchTree} 类的测试程序与 \texttt{remove} 函数的实现细节。本文将介绍测试程序的目的、\texttt{remove} 函数的设计与实现,以及测试结果与分析。

\section{测试程序概述}

测试程序的主要功能是验证二叉搜索树 (\texttt{BinarySearchTree}) 中的删除操作是否正常运行。测试程序通过以下步骤逐步验证了删除操作在不同情况下的行为:

\begin{enumerate}
    \item \textbf{插入操作测试}:先插入一系列整数元素,构建一个具有多个节点的二叉搜索树。
    \item \textbf{删除叶子节点}:删除一个叶子节点,以测试删除操作对树的最简单结构是否正确处理。
    \item \textbf{删除具有一个子节点的节点}:删除一个具有单个子节点的节点,以测试删除操作能否正确重新连接该节点的子树。
    \item \textbf{删除具有两个子节点的节点}:删除一个具有两个子节点的节点,测试删除操作在复杂结构中的表现,特别是最小节点的替换。
    \item \textbf{删除不存在的元素}:尝试删除树中不存在的节点,以验证 \texttt{remove} 函数是否可以正确处理无效删除请求。
    \item \textbf{清空树}:通过 \texttt{makeEmpty} 方法清空树,以确保内存释放操作的正确性。
\end{enumerate}

测试程序中的每一步都打印树的当前结构,以便观察每次删除后的树的状态,从而检查删除操作的正确性。

\section{\texttt{remove} 函数的设计与实现}

\subsection{\texttt{remove} 函数简介}

\texttt{remove} 函数的主要作用是从二叉搜索树中删除指定的元素。删除操作分为以下几种情况:

\begin{itemize}
    \item \textbf{删除叶子节点}:如果目标节点是叶子节点,即没有子节点,直接删除该节点。
    \item \textbf{删除仅有一个子节点的节点}:如果目标节点仅有一个子节点,用该子节点替换目标节点。
    \item \textbf{删除具有两个子节点的节点}:如果目标节点有两个子节点,需找到目标节点右子树的最小节点(即中序后继节点)来替代目标节点,然后在右子树中删除该最小节点。
\end{itemize}

\subsection{\texttt{remove} 函数的实现}


\subsection{代码解释}

\begin{enumerate}
    \item \textbf{空指针检查}:如果 \texttt{t} 为 \texttt{nullptr},表示树为空或未找到要删除的节点,直接返回。
    \item \textbf{递归查找节点}:通过比较元素值 \texttt{x} 和当前节点 \texttt{t} 的值,决定向左子树或右子树继续查找。
    \item \textbf{删除逻辑}:
    \begin{itemize}
        \item \textbf{两个子节点的情况}:找到右子树中最小的节点 \texttt{min},用 \texttt{min} 替换当前节点 \texttt{t} 的位置。然后删除 \texttt{min} 节点的原始位置。
        \item \textbf{一个或零个子节点的情况}:如果目标节点仅有一个子节点或没有子节点,则直接用其唯一的子节点或 \texttt{nullptr} 替代该节点。
    \end{itemize}
\end{enumerate}

\subsection{\texttt{Min} 函数}

在 \texttt{remove} 函数中,用 \texttt{Min} 函数来找到右子树中的最小节点。


这个 \texttt{Min} 函数使用循环遍历来查找最左侧节点,并将其从原树中移除。

\section{测试结果}

以下是各项测试的预期结果:

\begin{enumerate}
    \item \textbf{插入节点测试}:成功构建二叉搜索树,树的结构正确。
    \item \textbf{删除叶子节点}:删除叶子节点后,树的结构保持有效,未出现不一致或错误的链接。
    \item \textbf{删除一个子节点的节点}:成功删除指定节点,并将该节点的子节点正确连接到父节点。
    \item \textbf{删除两个子节点的节点}:删除节点后,右子树中最小节点替换删除的节点,树的结构保持有效。
    \item \textbf{删除不存在的节点}:删除操作不会对树产生影响,也不会抛出异常。
    \item \textbf{清空树}:所有节点均被释放,树最终为空。
\end{enumerate}

在实际运行中,程序的输出应与预期一致,表明 \texttt{remove} 函数在各种情况下均能够正确执行删除操作。

\section{结论}

本次测试验证了 \texttt{BinarySearchTree} 类中 \texttt{remove} 函数的正确性和健壮性。在不同的删除场景下,\texttt{remove} 函数均能按预期工作,并正确调整树的结构。通过这次测试,我们确认 \texttt{remove} 函数能够处理二叉搜索树的删除操作,包括删除叶子节点、单子节点和双子节点的情况。

\end{document}
