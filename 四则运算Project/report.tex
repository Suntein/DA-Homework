\documentclass[12pt]{ctexart}
\usepackage{amsmath,amssymb,geometry}
\usepackage{listings}
\usepackage{xcolor}
\geometry{a4paper, margin=1in}

\lstset{
    basicstyle=\ttfamily\small,
    keywordstyle=\color{blue}\bfseries,
    commentstyle=\color{gray},
    stringstyle=\color{red},
    frame=single,
    numbers=left,
    numberstyle=\tiny,
    breaklines=true,
    showstringspaces=false,
}

\title{四则运算计算器Project}
\author{}
\date{}

\begin{document}

\maketitle

\section*{功能概述}
本项目实现了一个表达式计算器,其主要功能包括:
\begin{enumerate}
    \item 支持将中缀表达式转换为后缀表达式(逆波兰表达式)。
    \item 支持计算后缀表达式的值。
    \item 识别非法表达式,包括括号不匹配、连续运算符、非法的表达式开头或结尾,以及除数为零等情况。
    \item 支持负数运算(如 $1 + -2.1$ 是合法的,但 $1++2.1$ 是非法的)。
    \item 支持科学计数法(如 $-1+2e2$ 是合法的)。
    \item 当出现非法行为时会标注非法类型:如divided by zero or Invalid Expression。
\end{enumerate}

\section*{功能详细说明}

\subsection*{1. 非法表达式检测}
代码实现了以下几种非法表达式的检测:
\begin{itemize}
    \item \textbf{括号不匹配}:使用栈操作检查括号匹配情况。如果右括号多于左括号,或表达式结束时栈中仍有未匹配的左括号,则表达式非法。
    \item \textbf{连续运算符}:判断相邻字符是否均为运算符(如 $1++2$ 或 $1+-+2$),如果是,则表达式非法。
    \item \textbf{非法开头或结尾}:表达式不能以运算符开头(如 $+1-2$)或以运算符结尾(如 $1+$)。
    \item \textbf{除数为零}:在后缀表达式计算时,检测除法操作的分母是否为零,若是,则抛出异常。
\end{itemize}

\subsection*{2. 支持负数}
负数的处理通过上下文判断实现:
\begin{itemize}
    \item 如果负号出现在表达式的开头,或在另一个运算符后面(如 $1 + -2$),则将其视为负数。
    \item 例如,表达式 $1 + -2.1$ 被正确识别为合法表达式,并转换为后缀表达式 $1\ 2.1\ -\ +$。
\end{itemize}

\subsection*{3. 支持科学计数法}
科学计数法的支持通过对 $e$ 操作符的特殊处理实现:
\begin{itemize}
    \item 在中缀转后缀阶段,$e$ 被识别为一个高优先级运算符。
    \item 在后缀表达式计算阶段,$a \cdot 10^b$ 的计算通过 $\texttt{pow(10, b)}$ 实现。
    \item 例如,$1 + 2e2$ 转换为后缀表达式 $1\ 2\ 2\ e\ +$,计算结果为 $201$。
\end{itemize}

\section*{测试用例}

\subsection*{合法表达式测试}

\begin{table}[h!]
\centering
\begin{tabular}{|l|l|l|}
\hline
\textbf{表达式} & \textbf{说明}                & \textbf{结果} \\ \hline
$1+2$           & 简单加法                    & $3$          \\ \hline
$1+-2$          & 包含负数                    & $-1$         \\ \hline
$-1+2$          & 表达式开头为负数            & $1$          \\ \hline
$3*(1+2)$       & 含括号的表达式              & $9$          \\ \hline
$1+2e2$         & 科学计数法                  & $201$        \\ \hline
$1+-2.1$        & 小数与负数                  & $-1.1$       \\ \hline
$(3+5)*2$       & 含括号的表达式              & $16$         \\ \hline
$1+2e-1$        & 科学计数法指数为负          & $1.2$        \\ \hline
$-2*(3+4)$      & 负数与括号                  & $-14$        \\ \hline
$1e3+2e-2$      & 科学计数法混合              & $1000.02$    \\ \hline
\end{tabular}
\caption{合法表达式测试用例}
\end{table}

\subsection*{非法表达式测试}

\begin{table}[h!]
\centering
\begin{tabular}{|l|l|l|}
\hline
\textbf{表达式} & \textbf{说明}          & \textbf{错误信息}      \\ \hline
$1++2$          & 连续运算符            & Invalid expression     \\ \hline
$(1+2$          & 括号不匹配            & Invalid expression     \\ \hline
$1+2)$          & 括号不匹配            & Invalid expression     \\ \hline
$+1-2$          & 以运算符开头          & Invalid expression     \\ \hline
$1/0$           & 除数为零              & Division by zero       \\ \hline
$1e$            & 科学计数法不完整      & Invalid expression     \\ \hline
$1+e2$          & 科学计数法不完整      & Invalid expression     \\ \hline
\end{tabular}
\caption{非法表达式测试用例}
\end{table}

\section*{核心实现逻辑}
\subsection*{1. 中缀转后缀表达式}
\begin{itemize}
    \item 遍历表达式字符串:
    \begin{itemize}
        \item 如果是数字或小数点,将其追加到当前数字字符串中。
        \item 如果是运算符,比较其优先级,将优先级较高或相等的运算符弹出到结果中。
        \item 如果是括号,左括号直接入栈,右括号将栈中操作符弹出直到匹配左括号。
    \end{itemize}
    \item 处理结束后,将剩余的数字和运算符依次加入结果。
\end{itemize}

\subsection*{2. 后缀表达式计算}
\begin{itemize}
    \item 遍历后缀表达式:
    \begin{itemize}
        \item 如果是数字,将其转换为浮点数并压入栈中。
        \item 如果是操作符,弹出栈顶两个操作数,执行相应运算,将结果重新压入栈。
        \item 如果检测到非法运算(如栈中元素不足或除数为零),抛出异常。
    \end{itemize}
\end{itemize}

\section*{总结}
本项目实现了一个功能完整的表达式计算器,支持负数、小数、科学计数法,并对常见非法表达式提供了有效的检测和错误处理机制。代码结构清晰,易于维护,可扩展性强。
\end{document}
